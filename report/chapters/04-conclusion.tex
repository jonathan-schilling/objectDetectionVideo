\chapter{Conclusion and Future Work}\label{chpt:conclusion+future}
\glsresetall

Finally a summary is provided and further ideas for project extension are given in the 
section~\ref{sec:future}.

\section{Conclusion}\label{sec:conclusion}

In this project a concept for an application was successfully developed and implemented. The 
application transforms a recording of the video game Valorant into a new representation that gives 
a better overview of the rounds for players and trainers and helps them to improve their skills.

The realization of this concept includes acquiring of data as well as the pre-pro\-ces\-sing of it with 
the objective to build a dataset. As training framework YOLOv5 from Ultralytics was chosen. In order 
to complete the training task it was necessary to get used to YOLOv5 and miscellaneous concepts in 
the field of the training of a neural network. Furthermore a script was implemented to transform the 
output of the neural network into an image, so that it is better understandable.

Challenging was to get enough labeled data because the preparation of it is very time consuming. 
Another problem is the limited hardware performance on private computers. These cause long 
training times and low performance networks have to be used which affect the results. But all in all 
satisfying results could be achieved.

\section{Future Work}\label{sec:future}

In following projects the learning task can be extended onto the detection of individual agents and 
abilities. With that information more helpful output formats can be developed. Additionally other 
learning techniques can be applied like the usage of time series processing algorithms. 
They could use available time information from videos for better detection. Furthermore it would be 
possible to use the output of neural networks, like that from this project, as input of other pattern 
recognition algorithms. These could be used to evaluate the gameplay in an advanced way.
